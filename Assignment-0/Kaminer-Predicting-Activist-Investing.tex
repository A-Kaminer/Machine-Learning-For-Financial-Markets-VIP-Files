\documentclass[11pt,a4paper]{article}
\usepackage[utf8]{inputenc}
\usepackage{amsfonts}
\usepackage{amsmath}
\usepackage{changepage}
\usepackage{graphicx}
\usepackage{indentfirst}
\usepackage{titlesec}
\usepackage[makeroom]{cancel}
\usepackage[margin=1in]{geometry}
\usepackage{titling}

\renewcommand*{\thesection}{\arabic{section}.}
\renewcommand*{\thesubsection}{\arabic{section}.\alph{subsection}.}
\renewcommand*{\thesubsubsection}{\arabic{section}.\alph{subsection}.\arabic{subsubsection}.}

% \titleformat{\section}
%     {\normalfont\fontsize{13}{13}\bfseries}{\thesection}{1em}{}


\title{VIP Research Proposal - Predicting Targets of Activist Investing}
\author{Andrew Kaminer}
\date{January 26, 2023}
\setlength{\droptitle}{-1in}
\vspace{-1in}

\begin{document}

\maketitle

\thispagestyle{empty}

\section{Introduction}

Activist investing is the act of an institutional investor buying significant
stock in a firm (often enough to require filing of a Chapter 13D) and attempting
to influence management, through peaceful and/or aggressive means, to make 
changes to company structure in order to extract higher shareholder value.
Brav, Jiang, Partnoy, and Thomas\cite{BravEtAll} state that activist investing
has existed to some extent since the 1980s. Interestingly, hey note that it has 
been ineffective in generating returns for shareholders in most forms. 
However, they show that activist hedge fund investing is in fact successful in 
improving returns for shareholders.

\section{Proposal}

The paper notes that target firms for activist investing usually follow a few
criteria. First, they usually have a low market cap relative to their assets.
Secondly, they usually have high levels of institutional ownership. This author
proposes that there are likely other signals of a firm that is a target for 
activist investing. Activist funds may already be aware of and actively 
monitoring these signals, or the signals may be unutilized and/or unintuitive.
For instance, how the management sells or buys shares may give an indication 
of whether or not a firm is a target

The market usually reacts positively to activist investing, creating abnormal
returns.\cite{BravEtAll} If a model could be constructed that predicts if a 
firm will be the target of an activist event in the next year, then the author
hypothesizes that a trading strategy capitalizing on that could produce outsized
returns. This model could also be used by prospective activist institutional
investors to identify targets.

\section{Models, Signals, Data, Etc}

An incomplete list of possible signals includes: abnormally high management
compensation, high management turnover, management trading large
quantities of shares in the company, high employee turnover, low valuation 
relative to what is normal in the sector, etc. There are many possible signals 
that could indicate that a firm is a good target for activist investing. 

The model would likely use some form of probit modelling, as the dependent 
variable is binary (firm will be a target or firm will not be a target). Thus,
the model will use some form of regression. The exact specifications of such a 
model need further investigations as the requirements of the model become 
better understood. 

The data could be collected through a similar methodology to the authors of the
original study\cite{BravEtAll}, scraping data from news sources (using natural 
language processing), or some preexisiting dataset. This data could be combined
with XBRL data to get a more complete picture of an activist event. 
Ultimately, it is the author's belief that the model is a worthwhile pursuit
that can provide value to academics, institutional, and retail investors alike.


\bibliographystyle{plain}
\bibliography{refs}

\end{document}
